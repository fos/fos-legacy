% Preview source code

%% LyX 1.6.4 created this file.  For more info, see http://www.lyx.org/.
%% Do not edit unless you really know what you are doing.
\documentclass[a4paper,english]{paper}
\usepackage[T1]{fontenc}
\usepackage[latin9]{inputenc}
\pagestyle{empty}
\setlength{\parskip}{\medskipamount}
\setlength{\parindent}{0pt}
\usepackage{color}
\usepackage{graphicx}
\usepackage{setspace}

\makeatletter

%%%%%%%%%%%%%%%%%%%%%%%%%%%%%% LyX specific LaTeX commands.
\newcommand{\lyxline}[1][1pt]{%
  \par\noindent%
  \rule[.5ex]{\linewidth}{#1}\par}
%% Because html converters don't know tabularnewline
\providecommand{\tabularnewline}{\\}

%%%%%%%%%%%%%%%%%%%%%%%%%%%%%% User specified LaTeX commands.
\usepackage{babel}
\usepackage{apacite}

\makeatother

\usepackage{babel}

\begin{document}
  \begin{singlespace}

    \noindent \begin{center}
    \textsf{\textbf{\LARGE }}%
    \begin{minipage}[t]{90mm}%
      \begin{singlespace}
        \noindent 
        \begin{center}
        \textsf{\textbf{\textsl{\textcolor{black}{\huge Identification of
Corresponding Tracks in Diffusion MRI Tractographies}}}}\textsf{\textbf{\LARGE \bigskip{}
}}
        \par
        \end{center}{\LARGE \par}

      \end{singlespace}

\noindent \begin{center}
{\Large Eleftherios Garyfallidis} {[}1, 2{]}
\par\end{center}

\noindent \begin{center}
{\Large Matthew Brett} {[}3{]}
\par\end{center}

\noindent \begin{center}
{\Large Vassilis Tsiaras} {[}3{]}
\par\end{center}

\noindent \begin{center}
{\Large George Vogiatzis} {[}1{]} 
\par\end{center}

\noindent \begin{center}
\&
\par\end{center}

\noindent \begin{center}
{\Large Ian Nimmo-Smith} {[}2{]}\bigskip{}

\par\end{center}

\noindent \lyxline{\normalsize}
\begin{enumerate}
\item \noindent \textsc{University of Cambridge}
\item \noindent \textsc{MRC Cognition and Brain Sciences Unit, Cambridge}
\item \noindent U\textsc{niversity of Berkeley, California}
\item \noindent \textsc{University of Crete, Greece\bigskip{}
}
\end{enumerate}
\end{minipage}

\par\end{center}{\LARGE \par}
\end{singlespace}

\newpage

\textbf{\Large }%
\begin{minipage}[t]{1\columnwidth}%
\textbf{\Large Key Research Question}{\Large \par}

\noindent \begin{flushleft}
{\Large How can we identify corresponding tracks between different
diffusion MR brain tractographies?} 
\par\end{flushleft}%
\end{minipage}{\Large \par}

\newpage

\textbf{\Large }%
\begin{minipage}[t]{1\columnwidth}%
\noindent \begin{flushleft}
\textbf{\Large Requirement to Create a Tractography}
\par\end{flushleft}{\Large \par}
\begin{itemize}
\item \begin{flushleft} a local diffusion model (e.g. simple tensor, Q-Ball)
\par\end{flushleft}
\item \begin{flushleft} a tractography algorithm (e.g. FACT, Runge-Kutta)
\par\end{flushleft}
\end{itemize}
\noindent \begin{flushleft} The quality and size of resulting tractographies may differ depending
on the choice of these options
\par\end{flushleft}
\begin{itemize}
\item \begin{flushleft} e.g. FACT 100,000; RK2 150,000)
\par\end{flushleft}
\end{itemize}
%
\end{minipage}{\Large \par}

\newpage

\textbf{\Large }%
\begin{minipage}[t]{1\columnwidth}%
\textbf{\Large Comparison of tractographies}{\Large \par}

{\large A: Brain 1+ Simple Tensor + FACT}{\large \par}

{\large B: Brain 1 + Simple Tensor + RK2}%
\end{minipage}{\Large \par}

\newpage

\textbf{\Large }%
\begin{minipage}[t]{1\columnwidth}%
\textbf{\Large Data compression}{\Large \par}

These large datasets take significant computational resources, in
particular long processing times. We have developed geometric techniques
which reduce the number of points on the tracks of a tractography
by a factor of 8. More points are retained where the track has high
curvatue and fewer are needed where the track has low curvature. The
time performance exceeds that of information theory based techniques
such as \emph{minimum description length} {[}1{]}.%
\end{minipage}{\Large \par}

\newpage

%
\begin{minipage}[t]{1\columnwidth}%
\item \textbf{\Large Track Metrics}{\Large \par}

Metrics are needed to measure that similarity of two tracks, whether
they are in the same or different tractographies. We have used the
mean average minimum distance metric (MAM) {[}2{]}:

$$\mathrm{MAM}(A,B) = \frac{1}{2}\Big[\frac{1}{\#A}\sum_{p\in A}\min_{q\in B} |p-q| + \frac{1}{\#B}\sum_{q\in B}\min_{p\in A} |q-p|\Big]$$%
\end{minipage}

\newpage

%
\begin{minipage}[t]{1\columnwidth}%
\textbf{\Large Corresponding Tracks}{\Large \par}

We can pick tracks of interest in one brain, and use this metric to
identify corresponding tracks in other brains quickly and accurately.%
\end{minipage}

\newpage

%
\begin{minipage}[t]{1\columnwidth}%
\textbf{\Large Adding Prior Knowledge }{\Large \par}

We can use a range of sources of prior knowledge
\begin{itemize}
\item white matter atlases {[}3{]}
\item using experts to label regions of the brain e.g. {[}4{]}
\item `picking' and expanding tracks to create ad hoc bundles 
\end{itemize}
%
\end{minipage}

\newpage

%
\begin{minipage}[t]{1\columnwidth}%
\textbf{\Large Conclusion}{\Large \par}

The correspondence problem has been solved in a manner which is supported
by high speed algorithms and high quality visualisation tools.%
\end{minipage}

\newpage

%
\begin{minipage}[t]{1\columnwidth}%
We are developing Python software
\begin{itemize}
\item \textsc{DiPy:} to support diffusion imaging analysis - now available
in open source
\item \textsc{Fos: }A 3d engine that supports multiple simultaneous visualisations 
\end{itemize}
%
\end{minipage}

\newpage

%
\begin{minipage}[t]{1\columnwidth}%
\textbf{\Large References and Credits}{\Large \par}

{[}1{]} Lee = Minimum description length

{[}2{]} Corouge = 

{[}3{]} Mori = atlas

{[}4{]} PBC

{[}3{]} Demiralp - Boy's surface colormap%
\end{minipage}
\end{document}
